\documentclass[12pt,titlepage]{article}

\usepackage{graphics}
\usepackage{pdflscape}
\usepackage{array}
\usepackage{color}
\usepackage{colortbl}
\usepackage{xcolor}
\usepackage{url,amsfonts,epsfig}
\usepackage[applemac]{inputenc} %comando per le lettere accentate se usate mac  
%\usepackage[T1]{fontenc}
%\usepackage[utf8]{inputenc}
\usepackage[english]{babel}
%\usepackage[latin1]{inputenc} % comando per le lettere accentate se usate pc  
\usepackage[pagebackref]{hyperref}
\hypersetup{
colorlinks=false,
allbordercolors=white
}

\begin{document}
%Code for title page
\begin{titlepage}
\centering
\includegraphics[width=0.4\textwidth]{Logos/LogoPolimi}\par
	{{Politecnico di Milano} \par}
	{{A.A. 2017/2018} \par}
	\vspace{1.5cm}
	\includegraphics[width=0.9\textwidth]{Logos/LogoTravlendar}\par
	%{\Large{\textsc{{{\color{red}{ \textbf{T}rav}}lendar{\color{red}{+}}}} \\ 
		%Software Engineering 2} \par}f
	\vspace{1.5cm}
	{\Huge \textbf {DD}\par}
	{ \textbf{Design Document} \par}
	\vspace{1.5cm}
	{\Large\itshape Sara Pid\'o  }{\Large   {  894744}\par}
	{\Large\itshape Chiara Plizzari }{\Large   {  893901}\par}
	{\Large\itshape Giuseppe Severino }{\Large   {  898458}\par}
	\vspace{2cm}
	\vfill
	% Bottom of the page
	%{\large Document version: 3.1\par}
\end{titlepage}

\newpage\null\thispagestyle{empty}\newpage

\pagenumbering{roman}

%%%% Opzione per interlinea 2
%%%\baselineskip 18pt

%\maketitle

\tableofcontents
%%\listoffigures
%%\listoftables

\pagebreak

\section{Introduction} \label{introduzione}
\pagenumbering{arabic}
\subsection{Purpose}
With the Design Document we would like to make the idea of Travlendar+ application more precise and more detailed.
In particular the main goal of the DD is  to describe the system in terms of architectural design choices.
It is written in particular for developers to help them to identify the architectural styles, the design patterns, the main components and their interfaces and, last but not least, the runtime behaviour.


\subsection{Scope}
The system aims to provide a complete calendar to users which are also helped to find the best route to reach their meetings and their appointments. 
Users can insert their meetings in order to have an agenda organized in a perfect way: they can see their trips of the day, their itineraries between appointments. Users can choose between different travel options basing on distances, travel time, cost.
The system provides some features to personalize the application in order to please users. In fact they can specify lunch time, they can activate or deactivate some travel means, they can also  choose for instance to minimize the walking distance or the carbon footprint. 
With this new application, people and their smartphones can have a lot of appointments in different locations without having the concern about plan the way to reach them. Obviously if two meetings overlap or if it is not possible to reach one, the system will advice users.

\subsection{Definitions, Acronyms, Abbreviations}
\subsection{Definitions} TODO CHECK AND CHANGE --> COPIED FROM RASD
\begin{itemize}
\item Visitor: a person that is not registered yet, but has the access to the application’s information.
\item Registered user: a person that is logged in the system and can create meetings.
\item Activity: an event that happens in the real world and that could be a meeting or a break.
\item Meeting: an activity among the registered user and other people. It can be created, modified and deleted by the meeting’s creator.
\item Break: an activity that a registered user can insert in order to manage it in a customizable way.
\item Trip: it indicates the route and the travel means chosen, based on user’s preferences.
\item Location: fixed place where a user stands or where he/she has to attend a meeting.
\item Global preferences: they are global attributes that registered users can modify and those are valid for all trips (i.e. minimize carbon footprint).
\item Creation screen: the screen of the application in which the registered user create a meeting or a break and enters its related details.
\item Blocked travel means: it is a travel means that the user has selected as unwanted.
\item Warning: a message directed to the user that arrives to him in form of a notification and can be shown on the application screen. It is generated by the system when there are some impediments for a trip (bad weather, strikes, traffic...) or when there are some problems (invalid data during the registration process or during the creation of an activity etc).
\end{itemize}
\subsection{Acronyms}
\begin{itemize}
\item DD: design document;
\item RASD: requirements analysis and specification document;
\item API: application programming interface;
\end{itemize}
\subsection{Abbreviations}

\subsection{Reference Documents}
\begin{itemize}
\item RASD;
\item Specification Document;
\item Example of DD of previous years;
\end{itemize}
\subsection{Document Structure}
This document is structured as follows:
\paragraph{Section 1: Introduction}
In this section it is described the purpose and the main goals  of the document giving a general description.
\paragraph{Section 2: Architectural Design}
It gives a general view on how the architecture of Travlendar+ should be showing architectural choices, styles and patterns.
\paragraph{Section 3: Algorithm Design}
In this part we include the most critical and relevant parts via algorithms.
\paragraph{Section 4: User Interface Design}
This section provides an overview on how the user will see the application through the mockups and UX and BCE diagrams.
\paragraph{Section 5: Requirements Traceability}
It explains how requirements defined in the RASD must be mapped to the design elements of the application.
\paragraph{Section 6: Implementation, Integration and Test Plan}
This part will include the order of implementation of subcomponents and to integrate them. Moreover it will include our plan to test this integration.
\paragraph{Section 7: Effort spent}
Here are reported the information about the hours of work spent by each member of the group by doing this project.
\paragraph{Section 8: References}

\section{Architectural Design}
\subsection{Overview}
\subsection{Component View}
\subsection{Deployment View}
\subsection{Runtime View}
\subsection{Component Interfaces}
\subsection{Selected Architectural Styles and Patterns}
\subsection{Other Design Decisions}

\section{Algorithm Design}

\section{User Interface Design}

\section{Requirements Traceability}

\section{Implementation, Integration and Test Plan}

\section{Effort Spent}

\section{References}
\end{document}